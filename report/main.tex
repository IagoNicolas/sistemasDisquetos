\documentclass[a4paper,12pt]{article}
\usepackage[utf8]{inputenc}
\usepackage{amsfonts}
\usepackage{hyperref}
\usepackage{listings}
\usepackage{amssymb}
\usepackage{amsmath}
\usepackage[left=1.6cm, right=1.6cm, top=2cm, bottom=2cm]{geometry}

\lstdefinestyle{CStyle}{
  basicstyle=\ttfamily,
  commentstyle=\color{green},
  keywordstyle=\color{blue},
  stringstyle=\color{red},
}

\title{Sistema Criptográfico RSA - Teoria e Aplicação}
\author{Iago Nicolas Lopes Lima}
\date{\today}

\begin{document}

\maketitle

\section*{Resumo}
Este relatório apresenta uma análise detalhada do sistema criptográfico RSA, abordando sua teoria fundamental e aplicação prática. Desenvolvido em 1977 por Ronald Rivest, Adi Shamir e Leonard Adleman, o RSA se destaca como um dos métodos mais seguros e amplamente utilizados na criptografia assimétrica. O trabalho explora os fundamentos teóricos do RSA, suas aplicações práticas, e oferece uma implementação em Python. Os principais resultados mostram que, embora o RSA seja eficaz na proteção de dados, ele enfrenta desafios relacionados ao desempenho e à gestão de chaves. A análise detalha esses desafios e propõe possíveis melhorias, consolidando o RSA como uma ferramenta essencial na segurança da informação.

\section*{Introdução/Motivação}
A criptografia desempenha um papel vital na proteção das informações na era digital. Com a crescente quantidade de dados sendo transmitidos e armazenados eletronicamente, a segurança dessas informações tornou-se uma prioridade absoluta. O algoritmo RSA, desenvolvido por Ronald Rivest, Adi Shamir e Leonard Adleman em 1977, é um dos marcos mais importantes na evolução da criptografia. Este projeto tem como objetivo explorar profundamente a teoria por trás do RSA e avaliar sua aplicação prática em contextos modernos de segurança da informação.

\subsection*{Objetivos}
Os objetivos principais deste projeto incluem:
\begin{itemize}
    \item Explorar e entender a teoria fundamental do sistema criptográfico RSA.
    \item Implementar o algoritmo RSA em um ambiente de programação controlado.
    \item Avaliar a eficácia do RSA na criptografia e descriptografia de dados.
    \item Identificar desafios e limitações na aplicação prática do RSA, sugerindo possíveis melhorias.
\end{itemize}

\subsection*{Requisitos}
Para a execução deste projeto, são necessários alguns pré-requisitos:
\begin{itemize}
    \item Conhecimento básico de teoria dos números e álgebra, especialmente conceitos relacionados a números primos e fatoração.
    \item Ferramentas de programação, preferencialmente Python, devido à sua simplicidade e ampla gama de bibliotecas matemáticas.
    \item Dados de teste para verificar a eficácia da criptografia e descriptografia.
\end{itemize}

\subsection*{Roteiro}
O projeto segue um roteiro estruturado em várias etapas, que incluem:
\begin{enumerate}
    \item Revisão de literatura sobre criptografia assimétrica e o algoritmo RSA.
    \item Estudo aprofundado da teoria matemática por trás do RSA, incluindo números primos, fatoração e aritmética modular.
    \item Implementação prática do RSA utilizando Python.
    \item Realização de testes de criptografia e descriptografia com diferentes tipos e tamanhos de dados.
    \item Análise dos resultados obtidos, identificando desafios e propondo melhorias.
    \item Conclusões e recomendações finais sobre a aplicação do RSA.
\end{enumerate}

A motivação para este projeto é clara: em um mundo cada vez mais digital, a segurança da informação é crucial. O RSA, como um dos algoritmos mais confiáveis e robustos, oferece uma excelente oportunidade para explorar e entender os mecanismos subjacentes à criptografia assimétrica. Além disso, este estudo contribui para o desenvolvimento contínuo de práticas seguras de comunicação e armazenamento de dados.

\section*{Fundamentação Teórica}

O RSA é um algoritmo de criptografia assimétrica que utiliza duas chaves distintas: uma chave pública para criptografar dados e uma chave privada para descriptografá-los. A segurança do RSA baseia-se na dificuldade de fatorar grandes números primos, um problema matemático que ainda não possui uma solução eficiente para números extremamente grandes.

\subsection*{Teoria dos Números}
A teoria dos números é uma área fundamental da matemática que estuda as propriedades dos números inteiros. Dois conceitos essenciais para o RSA são os números primos e a fatoração de inteiros.

\subsubsection*{Números Primos}
Números primos são números naturais maiores que 1 que não podem ser formados pela multiplicação de dois números naturais menores. Exemplos incluem 2, 3, 5, 7, 11, 13, e assim por diante. A importância dos números primos na criptografia RSA reside no fato de que a multiplicação de dois números primos grandes é fácil, mas a fatoração do produto resultante é extremamente difícil.

\subsubsection*{Fatoração de Inteiros}
Fatorar um número inteiro significa expressá-lo como um produto de números primos. Por exemplo, 15 pode ser fatorado em 3 e 5. A dificuldade de fatorar números grandes é a base da segurança do RSA. Se um adversário consegue fatorar o produto de dois números primos grandes usados na chave RSA, ele pode comprometer a segurança do sistema.

\subsubsection*{Aritmética Modular}
A aritmética modular é um sistema de aritmética para inteiros, onde os números "reiniciam" após atingir um certo valor, denominado módulo. A operação mais comum na aritmética modular é o cálculo do resto de uma divisão inteira, conhecida como operação de módulo. Por exemplo, \( 17 \mod 5 = 2 \), porque 17 dividido por 5 é 3 com um resto de 2.

\subsection*{O Algoritmo RSA}
O algoritmo RSA envolve três etapas principais: geração de chaves, criptografia e descriptografia.

\subsubsection*{Geração de Chaves}
A geração de chaves no RSA envolve os seguintes passos:
\begin{enumerate}
    \item Escolha de dois números primos grandes, \( p \) e \( q \).
    \item Cálculo de \( n = p \times q \), onde \( n \) é utilizado como o módulo tanto para a chave pública quanto para a chave privada.
    \item Cálculo do totiente de \( n \), \( \phi(n) = (p-1)(q-1) \).
    \item Escolha de um número inteiro \( e \) tal que \( 1 < e < \phi(n) \) e \( e \) seja coprimo a \( \phi(n) \). O valor \( e \) torna-se o expoente da chave pública.
    \item Cálculo do expoente da chave privada \( d \), que é o inverso multiplicativo de \( e \) módulo \( \phi(n) \), ou seja, \( d \times e \equiv 1 \pmod{\phi(n)} \).
\end{enumerate}

\subsubsection*{Justificação dos Teoremas Utilizados}
A segurança e funcionamento do RSA dependem de vários teoremas fundamentais da teoria dos números. Abaixo, apresentamos os teoremas utilizados e suas justificações.

\textbf{Teorema da Aritmética Modular}
O Teorema da Aritmética Modular, que é a base para o cálculo de \( e \) e \( d \), afirma que para qualquer número inteiro \( a \) e qualquer módulo \( m \), existe um número inteiro \( k \) tal que:
\[ a^k \mod m \]
A aritmética modular é usada para garantir que as operações de criptografia e descriptografia permaneçam dentro de um intervalo manejável de valores inteiros.

\textbf{Pequeno Teorema de Fermat}
O Pequeno Teorema de Fermat afirma que se \( p \) é um número primo e \( a \) é um número inteiro tal que \( p \) não divide \( a \), então:
\[ a^{p-1} \equiv 1 \pmod{p} \]
Este teorema é utilizado para provar que a exponenciação modular usada no RSA é correta. Ele garante que:
\[ a^{\phi(n)} \equiv 1 \pmod{n} \]
quando \( a \) é coprimo a \( n \).

\textbf{Teorema de Euler}
O Teorema de Euler generaliza o Pequeno Teorema de Fermat. Ele afirma que se \( n \) é um número inteiro positivo e \( a \) é um inteiro coprimo a \( n \), então:
\[ a^{\phi(n)} \equiv 1 \pmod{n} \]
onde \( \phi(n) \) é a função totiente de Euler, que conta o número de inteiros positivos menores que \( n \) que são coprimos a \( n \). Este teorema é crucial para a definição da chave privada \( d \), garantindo que a operação de descriptografia reverta corretamente a operação de criptografia.

\textbf{Algoritmo de Euclides Estendido}
O Algoritmo de Euclides Estendido é utilizado para encontrar o inverso multiplicativo de \( e \) módulo \( \phi(n) \). Este algoritmo é uma extensão do Algoritmo de Euclides, que encontra o máximo divisor comum (MDC) de dois inteiros. O Algoritmo de Euclides Estendido permite encontrar inteiros \( x \) e \( y \) tais que:
\[ ax + by = \text{MDC}(a, b) \]
No caso do RSA, ele é usado para encontrar \( d \) tal que:
\[ d \times e \equiv 1 \pmod{\phi(n)} \]
garantindo assim que \( d \) seja o inverso multiplicativo de \( e \) módulo \( \phi(n) \).

\subsection*{Criptografia e Descriptografia}

\subsubsection*{Criptografia}
Para criptografar uma mensagem \( m \):
\begin{enumerate}
    \item Converte-se a mensagem \( m \) em um número \( M \) tal que \( 0 \leq M < n \).
    \item Calcula-se o texto cifrado \( C \) utilizando a fórmula \( C = M^e \mod n \).
\end{enumerate}

\subsubsection*{Descriptografia}
Para descriptografar o texto cifrado \( C \):
\begin{enumerate}
    \item Calcula-se a mensagem \( M \) utilizando a fórmula \( M = C^d \mod n \).
    \item Converte-se o número \( M \) de volta para a mensagem original \( m \).
\end{enumerate}

\subsection*{Segurança do RSA}
A segurança do RSA depende da dificuldade de fatorar o produto de dois números primos grandes. Embora algoritmos eficientes de fatoração existam, eles não são práticos para números suficientemente grandes. Isso torna a criptografia RSA segura contra ataques de força bruta e fatoração direta.

\subsection*{Algoritmos Relacionados}
Além do RSA, existem outros algoritmos de criptografia assimétrica, como o ElGamal e o ECC (Elliptic Curve Cryptography). Cada um desses algoritmos possui suas próprias vantagens e desvantagens, dependendo da aplicação específica. Comparativamente, o RSA é amplamente utilizado devido à sua simplicidade e robustez.

\subsection*{Assinatura}
SHA-256 (Secure Hash Algorithm 256-bit) é uma função de hash criptográfica que produz um hash de 256 bits (32 bytes) a partir de uma mensagem de entrada de qualquer tamanho. Ele é utilizado para garantir a integridade e a autenticidade das mensagens.

\subsubsection*{Geração da assinatura}
\begin{enumerate}
    \item \textbf{Hashing da Mensagem:}
        \begin{itemize}
            \item A mensagem \(M\) é processada através do SHA-256 para produzir um hash \(H\);
            \item Este hash \(H\) é uma representação fixa e única da mensagem \(M\): \(H = SHA-256(M)\).
        \end{itemize}
    \item \textbf{Criptografia do Hash:}
        \begin{itemize}
            \item O hash \(H\) é então criptografado usando a chave privada \(d\) do remetente. O resultado é a assinatura digital \(S\): \(S = H^{d}\) mod \(n\).
        \end{itemize}
    \item \textbf{Verificação da Assinatura:}
        \begin{itemize}
            \item O destinatário calcula o hash \(H'\) da mensagem recebida \(M\) utilizando SHA-256: \(H' = SHA-256(M)\);
            \item A assinatura \(S\) é descriptografada usando a chave pública \(e\) do remetente para obter o hash \(H\): \(H = S^{e}\) mod \(n\);
            \item O destinatário compara o hash calculado \(H'\) com o hash descriptografado \(H\). Se \(H' = H\), a assinatura é válida.
        \end{itemize}
\end{enumerate}

\subsubsection*{Vulnerabilidades do RSA}
Vamos considerar um sistema RSA onde o usuário A tem uma chave pública composta por \( n \) e \( e \). Suponhamos os seguintes valores:
\begin{itemize}
    \item \( p = 61 \)
    \item \( q = 53 \)
    \item Portanto, \( n = p \times q = 61 \times 53 = 3233 \)
    \item O expoente público \( e = 17 \)
\end{itemize}
Considere também uma mensagem \( m = 65 \).

\subsubsection*{Como a chave de A é quebrada}
Primeiro, verificamos se \( \text{MDC}(m, n) \neq 1 \). Vamos usar a função de Máximo Divisor Comum (MDC).

\subsubsection*{Cálculo do MDC}
Para \( m = 65 \) e \( n = 3233 \):
\[
\text{MDC}(65, 3233)
\]
Usando o algoritmo de Euclides:
\begin{align*}
3233 &= 65 \times 49 + 48 \\
65 &= 48 \times 1 + 17 \\
48 &= 17 \times 2 + 14 \\
17 &= 14 \times 1 + 3 \\
14 &= 3 \times 4 + 2 \\
3 &= 2 \times 1 + 1 \\
2 &= 1 \times 2 + 0 \\
\end{align*}
Portanto, \( \text{MDC}(65, 3233) = 1 \).

Para fins de exemplo, suponhamos \( m = 3230 \):
\begin{align*}
3233 &= 3230 \times 1 + 3 \\
3230 &= 3 \times 1076 + 2 \\
3 &= 2 \times 1 + 1 \\
2 &= 1 \times 2 + 0 \\
\end{align*}
Então, \( \text{MDC}(3230, 3233) = 3 \).

\subsubsection*{Fatoração usando o MDC}
Encontramos um fator de \( n \), que é 3. Agora podemos encontrar o outro fator:
\[
\frac{n}{3} = \frac{3233}{3} = 1077.67
\]
Isso indica que \( 3 \) e \( 1077.67 \) são os fatores de \( n \). Podemos usar a mesma lógica para números mais apropriados onde \( \text{MDC}(m, n) \neq 1 \) resulta em fatores corretos.

\subsubsection*{Como falsificar a assinatura de A}
Após encontrar \( p \) e \( q \), podemos calcular a chave privada \( d \).

\subsubsection*{Calcular \( \phi(n) \)}
\[
\phi(n) = (p-1)(q-1) = (61-1)(53-1) = 60 \times 52 = 3120
\]

\subsubsection*{Encontrar \( d \)}
Calcular \( d \) tal que \( d \times e \equiv 1 \mod \phi(n) \):
\[
e = 17
\]
\[
\phi(n) = 3120
\]
Usando o algoritmo estendido de Euclides:
\begin{align*}
3120 &= 17 \times 183 + 9 \\
17 &= 9 \times 1 + 8 \\
9 &= 8 \times 1 + 1 \\
8 &= 1 \times 8 + 0 \\
\end{align*}
Portanto, \( d \equiv 2753 \mod 3120 \).

\subsubsection*{Falsificação de Assinatura}
Suponhamos que a mensagem \( m' = 89 \).
\begin{enumerate}
    \item Calcular o hash \( H \) de \( m' \) usando SHA-256 (simplificado como \( H = 89 \) para este exemplo).
    \item Criar a assinatura falsa \( S \):
    \[
    S = H^d \mod n
    \]
    \[
    S = 89^{2753} \mod 3233
    \]
\end{enumerate}

\subsubsection*{O sistema de decifragem funciona normalmente}
Mesmo se \( \text{MDC}(m, n) \neq 1 \), o processo de decifragem funciona normalmente.

\subsubsection*{Criptografia}
Para um texto cifrado \( C \):
\[
C = m^e \mod n
\]
\[
C = 65^{17} \mod 3233
\]
\[
C = 2790
\]

\subsubsection*{Decifragem}
Para decifrar:
\[
m' = C^d \mod n
\]
\[
m' = 2790^{2753} \mod 3233
\]
\[
m' = 65
\]
Portanto, a decifragem reverte corretamente para \( m \), provando que o sistema funciona mesmo com \( \text{MDC}(m, n) \neq 1 \).

\subsubsection*{Calcule a probabilidade de ocorrência dessa situação}
\subsubsection*{Probabilidade de Coprimos}
A probabilidade de dois números \( m \) e \( n \) serem coprimos é \( 1/\zeta(2) \), onde \( \zeta \) é a função zeta de Riemann.
\[
\zeta(2) = \frac{\pi^2}{6} \approx 1.64493
\]
Portanto, a probabilidade de \( \text{MDC}(m, n) = 1 \) é \( 1/\zeta(2) \approx 0.60793 \).

\subsubsection*{Probabilidade de Não Serem Coprimos}
A probabilidade de \( \text{MDC}(m, n) \neq 1 \):
\[
P(\text{MDC}(m, n) \neq 1) = 1 - 0.60793 \approx 0.39207
\]

\section*{Metodologia}
A metodologia para este projeto envolveu a implementação do algoritmo RSA, seguida por uma série de testes para verificar sua eficácia e desempenho. Abaixo estão as etapas detalhadas do projeto.

\subsection*{Verificação de Requisitos}
Os requisitos do sistema foram verificados através de uma série de testes que incluíam:

\begin{enumerate}
    \item \textbf{Teste de Correção}: Verificação se uma mensagem criptografada e, em seguida, descriptografada retorna à sua forma original.
    \item \textbf{Teste de Desempenho}: Medição do tempo de execução para criptografia e descriptografia de mensagens de diferentes tamanhos.
    \item \textbf{Teste de Segurança}: Avaliação da resistência do sistema contra tentativas de fatoração e ataques de força bruta.
\end{enumerate}

\subsection*{Resultados Esperados}
Esperava-se que o sistema RSA fosse capaz de criptografar e descriptografar mensagens de forma segura e eficiente, com tempos de execução razoáveis para mensagens de tamanho moderado. Além disso, o sistema deveria demonstrar robustez contra tentativas de ataques.

\subsection*{Ambiente de Desenvolvimento}
O ambiente de desenvolvimento para a implementação do RSA foi configurado em uma máquina local com as seguintes especificações:
\begin{itemize}
    \item Sistema Operacional: Fedora 40
    \item Processador: Intel Core i7-10710u
    \item Memória RAM: 64GB
    \item IDE: CLion 2024.1.2
    \item Padrão C: C99
\end{itemize}

\subsection*{Procedimento de Teste}
O procedimento de teste envolveu a criptografia e descriptografia de mensagens de diferentes tamanhos para avaliar a correção e o desempenho do algoritmo. As mensagens testadas variaram de pequenas frases a longos textos, incluindo dados binários.

\subsection*{Segurança e Gestão de Chaves}
A segurança e gestão de chaves são aspectos críticos na implementação do RSA. Foram adotadas práticas recomendadas para a geração segura de números primos e o armazenamento seguro das chaves privadas. A geração de números primos foi realizada utilizando a biblioteca Openssl, que oferece funções robustas para gerar números primos grandes com segurança. As chaves privadas e os componentes que a geram foram armazenadas em um formato não criptografado devido ao contexto de teste que temos.

\section*{Implementação do RSA}
A implementação do RSA foi realizada utilizando a linguagem de programação C, devido à sua simplicidade e vasta biblioteca de funções matemáticas.

\subsection*{Ferramentas Utilizadas}
\begin{itemize}
    \item Linguagem de programação C
    \item Bibliotecas: Openssl
\end{itemize}

\subsection*{Passos da Implementação}
\begin{enumerate}
    \item Geração de um par de chaves RSA;
    \item Salvamento das chaves em arquivos;
    \item Salvamento dos componentes das chaves em arquivos;
    \item Criptografia e descriptografia de dados;
    \item Assinatura e verificação de mensagens.
\end{enumerate}

\subsection*{Estrutura do Programa}
O programa é estruturado em várias funções para modularizar a funcionalidade, facilitando a manutenção e a clareza do código.

\subsubsection*{Definição de bibliotecas e Tratamento de Erros}
\begin{lstlisting}[style=CStyle]
    #include <stdio.h>
    #include <stdlib.h>
    #include <string.h>
    #include <openssl/rsa.h>
    #include <openssl/pem.h>
    #include <openssl/err.h>

    #define RSA_KEY_BITS 1024

    void handleErrors(void) {
        ERR_print_errors_fp(stderr);
        abort();
    }
\end{lstlisting}

\begin{itemize}
    \item \textbf{\lstinline[style=CStyle]{#include <openssl/rsa.h>}, \lstinline[style=CStyle]{#include <openssl/pem.h>} e \lstinline[style=CStyle]{#include <openssl/err.h>}}: Inclusão das bibliotecas necessárias do OpenSSL para RSA, PEM (formato de arquivo de chave) e tratamento de erros;
    \item \textbf{\lstinline[style=CStyle]{RSA_KEY_BITS 1024}}: Define o tamanho da chave RSA em bits;
    \item \textbf{\lstinline[style=CStyle]{handleErrors(void)}}: Função para lidar com erros do OpenSSL, imprimindo-os e abortando a execução do programa.
\end{itemize}

\subsubsection*{Geração do Par de Chaves RSA}
\begin{lstlisting}[style=CStyle]
    void generate_keypair(RSA **rsa_keypair) {
        BIGNUM *e = BN_new();
        RSA *rsa = RSA_new();

        if (!e || !rsa) { handleErrors(); }

        BN_set_word(e, RSA_F4);

        if (!RSA_generate_key_ex(rsa, RSA_KEY_BITS, e, NULL)) {
            handleErrors();
        }

        *rsa_keypair = rsa;

        BN_free(e);
    }
\end{lstlisting}

A função generate\_keypair é responsável por gerar o par de chaves RSA. Vamos detalhar os passos e os cálculos envolvidos:
\begin{enumerate}
    \item \textbf{Inicialização das Estruturas:}
      \begin{itemize}
        \item \textbf{\lstinline[style=CStyle]{BIGNUM *e = BN_new()} e \lstinline[style=CStyle]{RSA *rsa = RSA_new()}:} Cria novas estruturas BIGNUM e RSA;
        \item Estas estruturas são essenciais para armazenar os componentes das chaves RSA.
      \end{itemize}
    \item \textbf{Definição do Expoente Público:}
      \begin{itemize}
        \item \textbf{\lstinline[style=CStyle]{BN_set_word(e, RSA_F4)}}: Define o expoente público como 65537 (\lstinline[style=CStyle]{RSA_F4});
        \item O valor 65537 é escolhido porque é um número primo que facilita operações eficientes e seguras na criptografia.
      \end{itemize}
    \item \textbf{Geração das Chaves:}
      \begin{itemize}
        \item \textbf{\lstinline[style=CStyle]{RSA_generate_key_ex(rsa, RSA_KEY_BITS, e, NULL)}}: Gera o par de chaves RSA com o tamanho definido;
        \item Este processo envolve a escolha de dois números primos grandes \(p\) e \(q\), e o cálculo dos seguintes componentes:
            \begin{itemize}
                \item \textbf{Módulo \(n = p * q\):} Produto dos dois primos;
                \item \textbf{Totiente \(\phi(n) = (p - 1) * (q - 1)\)};
                \item \textbf{Chave pública \((e, n)\)}: Em que \(e\) é o expoente público e \(n\) é o módulo;
                \item \textbf{Chave privada \((d, n)\)}: Onde \(d\) é o inverso multiplicativo de \(e\) módulo \(\phi(n)\), calculado para satisfazer \(d * e \equiv 1\) (mod \(\phi(n))\).
            \end{itemize}
      \end{itemize}
\end{enumerate}
\subsubsection*{Salvamento das Chaves em Arquivos}
\begin{lstlisting}[style=CStyle]
    void save_keys(RSA *rsa_keypair) {
        FILE *pub_file = fopen("public_key.txt", "wb");
        if (!pub_file) { handleErrors(); }
        PEM_write_RSA_PUBKEY(pub_file, rsa_keypair);
        fclose(pub_file);

        FILE *priv_file = fopen("private_key.txt", "wb");
        if (!priv_file) { handleErrors(); }
        PEM_write_RSAPrivateKey(priv_file, rsa_keypair, NULL, NULL, 0, NULL, NULL);
        fclose(priv_file);
    }
\end{lstlisting}
A função save\_keys salva as chaves pública e privada em arquivos no formato PEM, que é um formato de arquivo que facilita a leitura e escrita de chaves.

\subsubsection*{Salvamento dos Componentes das Chaves}
\begin{lstlisting}[style=CStyle]
    void write_bn_to_file(FILE *file, const char *label, const BIGNUM *bn) {
        char *dec = BN_bn2dec(bn);
        if (!dec) { handleErrors(); }
        fprintf(file, "%s: %s\n", label, dec);
        OPENSSL_free(dec);
    }
    
    void save_key_components(RSA *rsa_keypair) {
        FILE *pub_file = fopen("public_key_components_hex.txt", "w");
        if (!pub_file) { handleErrors(); }
    
        const BIGNUM *n = RSA_get0_n(rsa_keypair);
        const BIGNUM *e = RSA_get0_e(rsa_keypair);
    
        write_bn_to_file(pub_file, "Modulus (n)", n);
        write_bn_to_file(pub_file, "Public Exponent (e)", e);
    
        fclose(pub_file);
    
        FILE *priv_file = fopen("private_key_components_hex.txt", "w");
        if (!priv_file) { handleErrors(); }
    
        const BIGNUM *d = RSA_get0_d(rsa_keypair);
        const BIGNUM *p = RSA_get0_p(rsa_keypair);
        const BIGNUM *q = RSA_get0_q(rsa_keypair);
        const BIGNUM *dmp1 = RSA_get0_dmp1(rsa_keypair);
        const BIGNUM *dmq1 = RSA_get0_dmq1(rsa_keypair);
        const BIGNUM *iqmp = RSA_get0_iqmp(rsa_keypair);
    
        write_bn_to_file(priv_file, "Modulus (n)", n);
        write_bn_to_file(priv_file, "Public Exponent (e)", e);
        write_bn_to_file(priv_file, "Private Exponent (d)", d);
        write_bn_to_file(priv_file, "Prime 1 (p)", p);
        write_bn_to_file(priv_file, "Prime 2 (q)", q);
        write_bn_to_file(priv_file, "Exponent 1 (dmp1)", dmp1);
        write_bn_to_file(priv_file, "Exponent 2 (dmq1)", dmq1);
        write_bn_to_file(priv_file, "Coefficient (iqmp)", iqmp);
    
        fclose(priv_file);
    }
\end{lstlisting}
Esta função salva os componentes das chaves em arquivos separados, permitindo uma análise detalhada dos valores individuais.

\subsubsection*{Criptografia e Descriptografia}
\begin{lstlisting}[style=CStyle]
    int main() {
        RSA *rsa_keypair = NULL;
        generate_keypair(&rsa_keypair);
        save_keys(rsa_keypair);
        save_key_components(rsa_keypair);
    
        const char *plaintext = "hello world!";
        int plaintext_len = strlen(plaintext);
    
        unsigned char *ciphertext = (unsigned char *)malloc(RSA_size(rsa_keypair));
        if (!ciphertext) { handleErrors(); }
    
        int ciphertext_len = RSA_public_encrypt(plaintext_len, (unsigned char *)plaintext, ciphertext, rsa_keypair, RSA_PKCS1_PADDING);
        if (ciphertext_len == -1) { handleErrors(); }
    
        unsigned char *decrypted_text = (unsigned char *)malloc(RSA_size(rsa_keypair));
        if (!decrypted_text) { handleErrors(); }
    
        int decrypted_len = RSA_private_decrypt(ciphertext_len, ciphertext, decrypted_text, rsa_keypair, RSA_PKCS1_PADDING);
        if (decrypted_len == -1) { handleErrors(); }
    
        printf("Plaintext: %s\n", plaintext);
        printf("Ciphertext: ");
        for (int i = 0; i < ciphertext_len; i++) {
            printf("%02x", ciphertext[i]);
        }
        printf("\n");
        printf("Decrypted text: %s\n", decrypted_text);
    
        unsigned char *signature = (unsigned char *)malloc(RSA_size(rsa_keypair));
        if (!signature) { handleErrors(); }
    
        const char *message = "Hello Bob!";
        int message_len = strlen(message);
    
        unsigned int signature_len;
        if (!RSA_sign(NID_sha256, (unsigned char *)message, message_len, signature, &signature_len, rsa_keypair)) {
            handleErrors();
        }
    
        int verified = RSA_verify(NID_sha256, (unsigned char *)message, message_len, signature, signature_len, rsa_keypair);
    
        (verified != 1) ? printf("Signature verification failed!\n") : printf("Signature verified. Message authenticated!\n");
    
        RSA_free(rsa_keypair);
        free(ciphertext);
        free(decrypted_text);
        free(signature);
    
        return 0;
    }
\end{lstlisting}

A função main realiza as operações de criptografia, descriptografia, assinatura e verificação. Vamos detalhar os cálculos envolvidos na criptografia e descriptografia:

\begin{enumerate}
    \item Criptografia:
        \begin{itemize}
            \item \textbf{RSA\_public\_encrypt:} Função que aplica a operação de criptografia usando a chave pública;
            \item \textbf{Fórmula:} \(C = M^{e}\) mod \(n\), onde \(M\) é a mensagem convertida em um numero, \(e\) é o expoente público e \(n\) o módulo;
            \item Essa operação eleva o numero \(M\) à potência \(e\) e em seguida aplica o módulo \(n\) resultando no texto cifrado \(C\).
        \end{itemize}
    \item Descriptografia:
        \begin{itemize}
            \item \textbf{RSA\_private\_decrypt}: Função que aplica a operação de descriptografia usando a chave privada;
            \item \textbf{Fórmula:} \(M = C^{d}\) mod \(n\), em que \(C\) é o texto cifrado e \(d\) é o expoente privado e \(n\) é o módulo;
            \item Esta operação eleva o texto cifrado \(C\) à potência \(d\) e em seguida aplica o módulo \(n\), resultando na mensagem original \(M\).
        \end{itemize}
    \item Assinatura e Verificação:
        \begin{itemize}
            \item \textbf{RSA\_sign}: Função que gera uma assinatura digital usando a chave privada;
            \item \textbf{RSA\_verify}: Função que verifica a assinatura digital usando a chave pública;
            \item As funções de assinatura e verificação utilizam o algoritmo de hash SHA-256 para garantir a integridade e autenticidade da mensagem.
        \end{itemize}
\end{enumerate}

\subsubsection*{Assinatura Digital usando SHA-256 e RSA}
A autenticação e a integridade da mensagem são garantidas combinando a função de hash SHA-256 com o algoritmo de criptografia RSA.

\textbf*{Geração de Assinatura Digital}
A função de hash SHA-256 (Secure Hash Algorithm 256-bit) é um dos algoritmos mais amplamente utilizados para garantir a integridade e a autenticidade de dados em segurança da informação. Abaixo, explicamos como a função de hash SHA-256 funciona em detalhes.

Uma função de hash criptográfica é um algoritmo que recebe uma entrada (ou mensagem) de qualquer tamanho e produz uma saída fixa (chamada de hash) de tamanho determinado. No caso do SHA-256, a saída é um valor hash de 256 bits (32 bytes).

Alguns motívos de nossa escolha é:
\begin{itemize}
    \item \textbf{Determinística:} A mesma mensagem de entrada sempre produzirá o mesmo hash de saída;
    \item \textbf{Rápida de Computar:} É rápido calcular o hash para qualquer mensagem;
    \item \textbf{Resistente a Pré-Imagem:} Dado um hash \(H\), é difícil encontrar uma mensagem \(M\) tal que hash(\(M\)) \(= H\);
    \item \textbf{Resistente a Segunda Pré-Imagem:} Dada uma mensagem \(M_{1}\), é difícil encontrar uma mensagem diferente \(M_{2}\) que tenha hash(\(M_{1}\)) \(=\) hash(\(M_{2})\);
    \item \textbf{Resistente a Colisão:} É difícil encontrar duas mensagens diferentes \(M_{1}\) e \(M_{2}\) que tenham o mesmo hash.
\end{itemize}

\textbf*{Etapas do SHA-256}
A função SHA-256 processa a mensagem de entrada em várias etapas para produzir o valor hash final. Aqui estão os passos principais:

\begin{enumerate}
    \item \textbf{Preprocessamento:}
        \begin{itemize}
            \item \textbf{Padding:} A mensagem de entrada é preenchida (ou "padded") para que seu comprimento em bits seja congruente a 448 (mod 512). Isso significa que a mensagem será preenchida para que o comprimento seja 64 bits a menos que um múltiplo de 512;
                \begin{itemize}
                    \item Adiciona-se um bit '1' ao final da mensagem;
                    \item Adicionam-se bits '0' até que o comprimento seja 448 (mod 512).
                \end{itemize}
            \item \textbf{Comprimento:} O comprimento original da mensagem é anexado ao final da mensagem preenchida, representado como um inteiro de 64 bits.
        \end{itemize}
    \item \textbf{Inicialização das Variáveis de Hash:}
        \begin{itemize}
            \item O SHA-256 utiliza oito variáveis de hash inicializadas com valores constantes específicos. Estas variáveis são:
                \begin{lstlisting}
                    h0 = 0x6a09e667
                    h1 = 0xbb67ae85
                    h2 = 0x3c6ef372
                    h3 = 0xa54ff53a
                    h4 = 0x510e527f
                    h5 = 0x9b05688c
                    h6 = 0x1f83d9ab
                    h7 = 0x5be0cd19
                \end{lstlisting}
        \end{itemize}
    \item \textbf{Processamento em Blocos:}
        \begin{itemize}
            \item A mensagem preenchida é dividida em blocos de 512 bits (64 bytes) para processamento;
            \item Cada bloco é então processado em uma série de 64 rodadas utilizando operações lógicas e aritméticas. As operações principais incluem:
                \begin{itemize}
                    \item \textbf{Funções Booleanas:} Ch, Maj, \(\sum\)0, \(\sum\)1, \(\sigma\)0, \(\sigma\)1;
                    \item \textbf{Constantes de Rodada:} SHA-256 usa uma série de constantes específicas para cada uma das 64 rodadas.
                \end{itemize}
        \end{itemize}
    \item \textbf{Transformação de Compressão:}
        \begin{itemize}
            \item Para cada bloco de 512 bits:
                \begin{itemize}
                    \item Expande-se o bloco em uma sequência de 64 palavras de 32 bits;
                    \item Inicializa-se oito variáveis de trabalho com os valores atuais das variáveis de hash;
                    \item Realiza-se uma série de 64 rodadas de operações de mistura que incluem adições, rotações e operações lógicas. Cada rodada usa uma constante específica e uma parte do bloco expandido;
                    \item Atualiza-se as variáveis de hash com os valores das variáveis de trabalho.
                \end{itemize}
        \end{itemize}
    \item Combinação dos Resultados:
        \begin{itemize}
            \item Após processar todos os blocos, os valores finais das variáveis de hash são concatenados para produzir o hash final de 256 bits.
        \end{itemize}
\end{enumerate}

\textbf*{Exemplo Simplificado}
Considere a mensagem "abc" para o exemplo simplificado a seguir:

\begin{enumerate}
    \item \textbf{Conversão para Binário:}
        \begin{itemize}
            \item "abc" em ASCII:
                \begin{itemize}
                    \item 'a' = 97 = 01100001;
                    \item 'b' = 98 = 01100010;
                    \item 'c' = 99 = 01100011;
                    \item "abc" = 01100001 01100010 01100011.
                \end{itemize}
        \end{itemize}
    \item \textbf{Padding:}
        \begin{itemize}
            \item Adicione '1':
            \begin{itemize}
                \item "01100001 01100010 01100011 1";
            \end{itemize}
            \item Inicialmente, a mensagem tem 24 bits. Precisamos adicionar 424 bits de padding \(448 - 24\) para totalizar 448 bits:
                \begin{itemize}
                    \item 01100001 01100010 01100011 1 00000000 00000000 ... (424 bits de '0')
                \end{itemize}
            \item Representamos o comprimento da mensagem original (24 bits) como um número binário de 64 bits e adicionamos ao final:
                \begin{itemize}
                    \item 01100001 01100010 01100011 1 00000000 ... 00000000 00000000 00000000 00000000 00000000 00000000 00000000 00000000 00000000 00011000
                \end{itemize}
        \end{itemize}
    \item \textbf{Inicialização e Processamento:}
        Agora que a mensagem tem exatamente 512 bits (64 bytes), o que é necessário para SHA-256, podemos realizar o processamento.
        \begin{itemize}
            \item Inicialize variáveis de hash (os oito valores citados acima);
            \item Divida a mensagem preenchida em blocos de 512 bits (já temos uma mensagem com esse tamanho);
            \item Realizam-se as 64 rodadas de operações de mistura para cada bloco, como o exemplo a seguir:
            \begin{lstlisting}
for i from 0 to 63:
    S1 := (e rightrotate 6) ^ (e rightrotate 11) ^ (e rightrotate 25)
    ch := (e and f) ^ ((not e) and g)
    temp1 := h + S1 + ch + k[i] + w[i]
    S0 := (a rightrotate 2) ^ (a rightrotate 13) ^ (a rightrotate 22)
    maj := (a and b) ^ (a and c) ^ (b and c)
    temp2 := S0 + maj

h := g
g := f
f := e
e := d + temp1
d := c
c := b
b := a
a := temp1 + temp2
            \end{lstlisting}
        \end{itemize}
    \item \textbf{Combinação dos Resultados:}
        \begin{itemize}
            \item Após processar o bloco de 512 bits, atualizam se as variáveis de hash:
            \begin{lstlisting}
h0 := h0 + a
h1 := h1 + b
h2 := h2 + c
h3 := h3 + d
h4 := h4 + e
h5 := h5 + f
h6 := h6 + g
h7 := h7 + h
            \end{lstlisting}
        \end{itemize}
    \item \textbf{Hash Final:}
        \begin{itemize}
            \item Finalmente, os valores das variáveis de hash (h0, h1, h2, h3, h4, h5, h6, h7) são concatenados para formar o hash final de 256 bits. Cada variável de hash é de 32 bits, totalizando 256 bits. O resultado é o hash SHA-256 da mensagem "abc".
        \end{itemize}
\end{enumerate}

\section*{Análise}
A análise dos resultados focou na comparação entre os resultados obtidos e os esperados, destacando os desafios encontrados durante o processo.

\subsection*{Correção dos Resultados}
Os testes de correção mostraram que o sistema RSA implementado conseguiu criptografar e descriptografar mensagens corretamente, confirmando a precisão da implementação. As mensagens criptografadas foram recuperadas integralmente após a descriptografia, demonstrando que o algoritmo foi implementado corretamente.

\subsection*{Desempenho}
Os testes de desempenho revelaram que, embora o RSA seja eficiente para mensagens curtas, o tempo de execução aumenta significativamente para mensagens maiores. Isso se deve à complexidade da aritmética modular com números grandes. A criptografia e descriptografia de mensagens longas resultaram em tempos de execução elevados, o que pode ser um fator limitante em aplicações práticas que exigem processamento em tempo real.

\subsection*{Segurança}
Os testes de segurança indicaram que o sistema RSA é robusto contra ataques de fatoração, desde que os números primos escolhidos sejam suficientemente grandes. No entanto, foi identificado que a geração de chaves pode ser um ponto fraco se não for realizada com cuidado, pois a escolha inadequada de primos pode comprometer a segurança. A análise de ataques de força bruta mostrou que, com números primos grandes, o RSA oferece uma segurança sólida contra tais ataques.

\subsection*{Desafios e Problemas}
Durante a implementação, foram encontradas dificuldades relacionadas à eficiência da geração de números primos grandes e à execução de operações aritméticas com números extremamente grandes. Além disso, a gestão de chaves revelou-se um desafio, destacando a necessidade de um armazenamento seguro e eficiente das chaves privadas. A geração de chaves RSA é um processo computacionalmente intensivo, e a escolha de números primos inadequados pode resultar em uma chave vulnerável.

\subsection*{Análise Comparativa}
Foi realizada uma análise comparativa entre o RSA e outros algoritmos de criptografia assimétrica, como o ElGamal e o ECC. A análise mostrou que o RSA oferece uma boa combinação de segurança e facilidade de implementação, mas enfrenta desafios de desempenho em comparação com algoritmos baseados em curvas elípticas (ECC), que podem oferecer uma segurança equivalente com chaves menores e operações mais rápidas.

\subsection*{Aplicações Práticas}
O RSA tem uma ampla gama de aplicações práticas, desde a segurança de comunicação em e-mails e navegadores da web até a assinatura digital de documentos e transações financeiras. No entanto, a aplicação do RSA em dispositivos com recursos limitados, como dispositivos móveis e IoT, pode ser desafiadora devido aos requisitos computacionais elevados.

\subsection*{Melhorias Propostas}
Para melhorar o desempenho do RSA, podem ser consideradas otimizações como a Exponenciação Modular Rápida e algoritmos eficientes de geração de primos. Além disso, a combinação do RSA com outros métodos de criptografia pode oferecer um equilíbrio entre segurança e eficiência. A implementação de hardware específico para operações aritméticas pode também melhorar o desempenho em aplicações críticas.

\section*{Conclusão}
O sistema criptográfico RSA, embora complexo, é uma ferramenta poderosa para a criptografia de dados. A implementação prática do RSA mostrou-se eficaz na proteção de informações, confirmando a teoria subjacente. No entanto, desafios relacionados ao desempenho e à gestão de chaves foram identificados.

\subsection*{Comentários sobre os Modelos Utilizados}
O modelo teórico do RSA é sólido, baseado em princípios matemáticos bem estabelecidos. A implementação prática demonstrou a viabilidade do RSA para aplicações reais, embora a eficiência possa ser um problema para grandes volumes de dados. A escolha cuidadosa de números primos e a geração segura de chaves são cruciais para a eficácia do RSA.

\subsection*{Possíveis Aplicações}
O RSA pode ser utilizado em diversas áreas, incluindo comunicação segura, assinatura digital e proteção de dados sensíveis. Sua capacidade de fornecer segurança robusta torna-o adequado para aplicações em setores financeiros, governamentais e de TI. No entanto, a aplicação do RSA em ambientes com restrições de recursos requer atenção especial aos requisitos computacionais.

\subsection*{Indicação de Aperfeiçoamento}
Para melhorar o desempenho do RSA, pode-se considerar o uso de otimizações como a Exponenciação Modular Rápida e algoritmos eficientes de geração de primos. Além disso, a combinação do RSA com outros métodos de criptografia pode oferecer um equilíbrio entre segurança e eficiência. A implementação de hardware específico para operações aritméticas pode também melhorar o desempenho em aplicações críticas. A adoção de práticas de gestão de chaves seguras é essencial para garantir a integridade e a confidencialidade dos dados.

\section*{Referências}
\begin{itemize}
    \item Rivest, R. L., Shamir, A., \& Adleman, L. (1978). A Method for Obtaining Digital Signatures and Public-Key Cryptosystems. Communications of the ACM, 21(2), 120-126.
    \item Menezes, A. J., van Oorschot, P. C., \& Vanstone, S. A. (1996). Handbook of Applied Cryptography. CRC Press.
    \item Trappe, W., \& Washington, L. C. (2006). Introduction to Cryptography with Coding Theory. Pearson.
    \item Stallings, W. (2016). Cryptography and Network Security: Principles and Practice. Pearson.
    \item Schneier, B. (1996). Applied Cryptography: Protocols, Algorithms, and Source Code in C. John Wiley \& Sons.
\end{itemize}

\end{document}